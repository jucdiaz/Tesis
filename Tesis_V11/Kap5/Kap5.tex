\chapter{Conclusiones y recomendaciones}\label{cap5}
\section{Conclusiones}

En este trabajo se propone una nueva clase de distribuciones y modelos de regresi\'{o}n para efectos fijos y mixtos para datos proporcionales inflados con ceros y/o unos, en la distribuci\'{o}n propuesta y los dos modelos de regresi\'{o}n, se integran tres diferentes parametrizaciones de la distribuci\'{o}n beta, adem\'{a}s a manera de distribuci\'{o}n alternativa de la beta se integra la distribuci\'{o}n simplex, estos modelos y distribuciones son muy flexibles, ya que en una sola distribuci\'{o}n o modelo de regresi\'{o}n se obtienen diferentes opciones de modelado.\\

La distribuci\'{o}n propuesta y en la que se basan todos los modelos de regresi\'{o}n, es la distribuci\'{o}n ZOIP, esta distribuci\'{o}n permite obtener diferentes distribuciones para datos proporciones inflados con ceros y/o unos, dicha distribuci\'{o}n permite ajustarse a datos que se encuentran inflados solo con ceros o solo con unos, incluso a datos proporcionales tradicionales, es decir, sin inflaciones. El modelo de regresi\'{o}n ZOIP es propuesto en este trabajo como el ajuste de un modelo de regresi\'{o}n con efectos fijos en los cuatro par\'{a}metros de distribuci\'{o}n ZOIP y es ajustado v\'{\i}a m\'{a}xima verosimilitud. Por \'{u}ltimo, se propone el modelo de regresi\'{o}n ZOIP mixto, el cual permite ajustar un modelo de regresi\'{o}n con efectos fijos en todos los par\'{a}metros de la distribuci\'{o}n ZOIP y tener en cuenta interceptos aleatorios normales en el par\'{a}metro de la media y la varianza, la estimaci\'{o}n de los par\'{a}metros fue realizada v\'{\i}a m\'{a}xima verosimilitud y mediante la aproximaci\'{o}n de la funci\'{o}n de verosimilitud, por medio de las diferentes alternativas de la cuadratura de Gauss-Hermite adaptativa multidimensional.\\
 
Fue propuesto un paquete de \proglang{R} llamado \pkg{ZOIP}, que permite obtener valores de la funci\'{o}n de densidad de probabilidad, funciones de distribuci\'{o}n acumulada y funci\'{o}n cuantil de la distribuci\'{o}n ZOIP, adem\'{a}s de generaci\'{o}n de valores aleatorios de dicha distribuci\'{o}n. Por otra parte, el paquete permite ajustar distribuciones ZOIP o modelos de regresi\'{o}n ZOIP, mediante algoritmos de optimizaci\'{o}n, como \code{nlimnb} o \code{optim}. Tambi\'{e}n, es posible ajustar los modelos de regresi\'{o}n ZOIP mixto, v\'{\i}a m\'{a}xima verosimilitud y la cuadratura de Gauss-Hermite adaptativa. El paquete \pkg{ZOIP}, tambi\'{e}n incluye en el ajuste de sus modelos algunas funciones de m\'{e}todo S3.\\

Se realizaron estudios de simulaci\'{o}n que permitieron ver la convergencia de los par\'{a}metros en el ajuste de la distribuci\'{o}n ZOIP, el modelo de regresi\'{o}n ZOIP y el modelo de regresi\'{o}n ZOIP mixto, adem\'{a}s esto permiti\'{o} concluir con que alternativas se encuentran las mejores estimaciones de los par\'{a}metros regresores de los modelos. En el modelo de regresi\'{o}n ZOIP mixto se encontr\'{o} que el hecho de utilizar la metodolog\'{\i}a \textit{pruning} ayuda a reducir los tiempos de ajuste del modelo en un 50\% cuando se utilizan grandes tama\~{n}os de muestra y el n\'{u}mero de puntos de la cuadratura de Gauss-Hermite es grande, adem\'{a}s se evidencio que el aumento del n\'{u}mero de puntos de la cuadratura afecta de manera positiva y en mayor proporci\'{o}n a los par\'{a}metros asociados a los interceptos aleatorios de los par\'{a}metros de la media y la varianza. Por \'{u}ltimo, se encontr\'{o} en todos los modelos que el aumento del tama\~{n}o de muestra es el mayor factor que hace que la estimaci\'{o}n de los par\'{a}metros regresores mejoren, considerablemente.\\

Por ultimo se realizaron aplicaciones a datos reales, para el ajuste de, distribuciones ZOIP, modelos de regresi\'{o}n ZOIP y modelos de regresi\'{o}n ZOIP mixtos sobre el porcentaje de utilizaci\'{o}n de una tarjeta de cr\'{e}dito asociada a una entidad financiera, en dicha aplicaci\'{o}n se evidencio que el ajuste de los modelos fueron satisfactorios y  fue posible encontrar diferentes factores que afectan al porcentaje de utilizaci\'{o}n de las tarjetas de cr\'{e}dito, lo que permitir\'{\i}a sacar conclusiones muy importantes para estrategias de mercadeo y fidelizaci\'{o}n de los clientes de la entidad financiera.


\section{Recomendaciones}

Como posibles trabajos futuros y recomendaciones se propone:

\begin{itemize}
	\item Extender la distribuci\'{o}n ZOIP a otras distribuciones para datos proporcionales no tomadas en cuenta en este trabajo, tales como la distribuci\'{o}n beta-rectangular y la distribuci\'{o}n LogitSep.
	\item Realizar un estudio de an\'{a}lisis de residuales para los par\'{a}metros estimados de los modelos de regresi\'{o}n ZOIP fijos y mixtos.
	\item Incluir un an\'{a}lisis de selecci\'{o}n de modelos en los modelos de regresi\'{o}n ZOIP fijos y mixtos.
	\item Incluir interceptos aleatorios en los par\'{a}metros asociados a la inflaci\'{o}n de la distribuci\'{o}n ZOIP, en el modelo de regresi\'{o}n ZOIP mixto, propuesto en este trabajo.
\item Incluir pendientes aleatorias en los diferentes par\'{a}metros de la regresi\'{o}n ZOIP.
	\item Extender los interceptos alea tiros en los par\'{a}metros de la media y la dispersi\'{o}n, a interceptos aleatorios no normales, o normales correlacionados.
	\item Incluir m\'{a}s funciones de m\'{e}todo S3 en el paquete \pkg{ZOIP}, tales como \code{plot}, \code{predict} \code{AIC}, \code{BIC}, entre otras.
	\item Realizar una comparaci\'{o}n entre la estimaci\'{o}n de los par\'{a}metros del modelo de regresi\'{o}n ZOIP mixto, por metodolog\'{\i}as bayesianas y utilizando m\'{a}xima verosimilitud y la cuadratura de Gauss-Hermite adaptativa.
		\item Mejorar la eficiencia de los algoritmos implementados en el paquete \pkg{ZOIP}, mediante la integraci\'{o}n del paquete \pkg{ZOIP} por las metodolog\'{\i}as implementadas en el paquete \pkg{RCPP}.
	\item Estudiar e implementar otras metodolog\'{\i}as de estimaci\'{o}n de los par\'{a}metros, nunca exploradas sobre los modelos de regresi\'{o}n mixtos para datos proporcionales inflados con ceros y/o con unos, tales como la metodolog\'{\i}a bayesiana INLA o los algoritmos propuestos por \cite{Ogden1}, sobre m\'{e}todos de reducci\'{o}n secuencial. 
	\item Evaluar m\'{a}s la simulaci\'{o}n del modelo de regresi\'{o}n ZOIP mixto propuesto, para las dem\'{a}s parametrizaciones de la distribuci\'{o}n ZOIP-beta y la distribuci\'{o}n ZOIP-simplex.
	\item Optimizar los algoritmos propuestos en el paquete \pkg{ZOIP}, para que el tiempo de estimaci\'{o}n los par\'{a}metros sean m\'{a}s cortos.
	\item Estudiar t\'{o}picos de los datos proporcionales inflados para datos proporcionales inflados con ceros y/o unos, como cuando m\'{a}s del 50\% de la muestra se encuentra inflada por ceros o por unos. se podr\'{\i}a evaluar la eficacia y convergencia de las funciones implementadas, en el paquete \pkg{ZOIP} mediante estudios de simulaci\'{o}n.
	\item Realizar mejoras continuas del paquete \pkg{ZOIP}, a medida que se vayan adquiriendo retroalimentaciones de los usuarios que utilizan el paquete.
	\item Extender los estudios de simulaci\'{o}n del modelo de regresi\'{o}n ZOIP mixto, cuando se tienen muestras no balanceadas.

\end{itemize}



\chapter{Introducci\'{o}n}


En el an\'{a}lisis de datos es usual estudiar variables aleatorias que den como respuesta datos proporcionales, derivados de c\'{a}lculos de tasas, porcentajes y razones; esta variable aleatoria se encuentra descrita por lo general en el intervalo (0,1) y se describe por lo general por la distribuci\'{o}n beta o la distribuci\'{o}n simplex \citep{Jorgensen1}, sobre la distribuci\'{o}n beta se han encontrado diferentes parametrizaciones, de acuerdo a la modificaci\'{o}n de sus par\'{a}metros, existe la parametrizaci\'{o}n original, la parametrizaci\'{o}n de \cite{Ferrari2} y la de \cite{Stasinopoulos2}, sin embargo, las anteriores distribuciones no tienen en cuentan que existen ocasiones donde este tipo de variables aleatoria dan respuestas de cero o de uno, Es por esto que otros autores como \cite{Ospina2} proponen una nueva clase de distribuci\'{o}n, por ejemplo la beta, que permiten incluir los valores de cero o uno, llamada distribuci\'{o}n beta inflada por ceros o unos.\\

El an\'{a}lisis de regresi\'{o}n sobre variables aleatorias infladas con ceros y/o unos, considera diferentes desaf\'{\i}os, uno de ellos es la inclusi\'{o}n en la modelaci\'{o}n de los par\'{a}metros que permiten describir la inflaci\'{o}n de los datos en ceros y en unos, por lo que autores como \cite{Ospina1} y \cite{Kosmidis1} han trabajado sobre el desarrollo de modelos de regresi\'{o}n para datos proporcionales inflados. El segundo desaf\'{\i}o es la estimaci\'{o}n de los par\'{a}metros regresores asociados a los cuatro par\'{a}metros de la distribuci\'{o}n que describe una variable aleatoria proporcional inflada con cero y/o unos, dicha estimaci\'{o}n puede ser realizada v\'{\i}a m\'{a}xima verosimilitud \citep{Ospina1} o mediante metodolog\'{\i}a bayesianas como la propuesta por \cite{Galvis1}, estas metodolog\'{\i}as deben ser resueltas computacionalmente.\\

Sobre los modelos de regresi\'{o}n mixtos para datos proporcionales inflados con ceros y/o unos, representan un desaf\'{\i}o mayor, entorno a la estimaci\'{o}n de los par\'{a}metros, es por eso que algunos autores como \cite{Usuga1}, \cite{Bonat2} \cite{Song1} y \cite{Stasinopoulos2} presentan modelos de regresi\'{o}n para datos proporcionales mixtos, sin incluir las inflaciones, la inclusi\'{o}n de los par\'{a}metros de inflaci\'{o}n m\'{a}s los efectos aleatorios hacen que la estimaci\'{o}n por m\'{a}xima verosimilitud no se resuelve computacionalmente f\'{a}cil, ni mucho menos anal\'{\i}ticamente, ya que requiere la soluci\'{o}n de integrales no cerradas asociadas a los efectos aleatorios, dichas integrales deben ser aproximadas por diferentes metodolog\'{\i}a como por ejemplo la cuadratura de Gauss-Hermite, introducido en los modelos lineales generalizados por \cite{Fahrmeir1}, o el m\'{e}todo de reducci\'{o}n secuencial \citep{Ogden1}, entre otros. El hecho de que la estimaci\'{o}n v\'{\i}a m\'{a}xima verosimilitud sea complicada, hace que otros autores como \cite{Galvis1} busquen una soluci\'{o}n alternativa dentro de las metodolog\'{\i}as bayesianas, MCMC.\\

A nivel computacional es posible mencionar que existen diferentes paquetes en el software \proglang{R} que describen los datos proporcionales, uno de ellos y el m\'{a}s com\'{u}n es el paquete \pkg{betareg} \citep{Zeileis1}, \citep{Ferrari1} y \citep{Simas1}, el cual incluye la distribuci\'{o}n beta, y la estimaci\'{o}n de modelos de regresi\'{o}n beta fijos y mixtos. \cite{Qiu1} implementan el paquete \pkg{simplexreg} el cual incluye la distribuci\'{o}n y modelo de regresi\'{o}n con efectos fijos de la distribuci\'{o}n simplex.\\

En este trabajo se incluye en una sola distribuci\'{o}n y en un solo modelo de regresi\'{o}n de efectos fijos y mixtos, las principales distribuciones y modelos de regresi\'{o}n para datos proporcionales inflados con ceros y/o unos, por lo que es creado la distribuci\'{o}n ZOIP (Zeros Ones Inflated Proporcional), los modelos de regresi\'{o}n ZOIP para efectos fijos y mixtos. La estimaci\'{o}n de los par\'{a}metros regresores de los modelos de regresi\'{o}n ZOIP, son realizados v\'{\i}a m\'{a}xima verosimilitud y la cuadratura de Gauss-Hermite. Adicional como no existe un paquete en \proglang{R} que re\'{u}na las principales distribuciones y regresiones de efectos fijos y mixtos para modelar los datos proporcionales inflados con ceros y/o unos, se implementa el paquete \pkg{ZOIP}, que permite generar y ajustar distribuciones y modelos de regresi\'{o}n para efectos fijos y mixtos para datos proporcionales inflados con ceros y/o unos.

\section{Organizaci\'{o}n del trabajo}

\chapter{Introducci\'{o}n}


En el an\'{a}lisis de datos es usual estudiar variables aleatorias que den como respuesta datos proporcionales, derivados de c\'{a}lculos de tasas, porcentajes y razones; esta variable aleatoria se encuentra descrita en el intervalo (0,1) y es caracterizada  por lo general por la distribuci\'{o}n beta o la distribuci\'{o}n simplex \citep{Jorgensen1}. Sobre la distribuci\'{o}n beta, se han encontrado diferentes parametrizaciones, de acuerdo a la modificaci\'{o}n de sus par\'{a}metros, existe la parametrizaci\'{o}n original, la parametrizaci\'{o}n de \cite{Ferrari2} y la de \cite{Stasinopoulos2}, sin embargo, las anteriores distribuciones no tienen en cuentan que existen ocasiones donde este tipo de variables aleatorias dan respuestas de cero o de uno, es por esto que otros autores como \cite{Ospina2} proponen una nueva clase de distribuci\'{o}n, por ejemplo la beta, que permiten incluir los valores de cero o uno, llamada distribuci\'{o}n beta inflada por ceros o unos.\\

El an\'{a}lisis de regresi\'{o}n sobre variables aleatorias infladas con ceros y/o unos, considera diferentes desaf\'{\i}os, uno de ellos es la inclusi\'{o}n en la modelaci\'{o}n de los par\'{a}metros que permiten describir la inflaci\'{o}n de los datos en cero y uno, por lo que autores como \cite{Ospina1} y \cite{Kosmidis1} han trabajado sobre el desarrollo de modelos de regresi\'{o}n para datos proporcionales inflados en ceros y/o unos. El segundo desaf\'{\i}o es la estimaci\'{o}n de los par\'{a}metros regresores asociados a los cuatro par\'{a}metros de la distribuci\'{o}n que describe una variable aleatoria para datos proporcionales inflada con cero y/o unos, dicha estimaci\'{o}n puede ser realizada v\'{\i}a m\'{a}xima verosimilitud \citep{Ospina1} o mediante metodolog\'{\i}as bayesianas como la propuesta por \cite{Galvis1}, estas metodolog\'{\i}as deben ser resueltas computacionalmente.\\

Los modelos de regresi\'{o}n mixtos para datos proporcionales inflados con ceros y/o unos, re\-pre\-sen\-tan un desaf\'{\i}o mayor, con respecto a la estimaci\'{o}n de los par\'{a}metros, es por eso que algunos autores como \cite{Usuga1}, \cite{Bonat2}, \cite{Song1} y \cite{Stasinopoulos2} presentan modelos de regresi\'{o}n para datos proporcionales mixtos, sin incluir las inflaciones en cero o uno, la inclusi\'{o}n de los par\'{a}metros de inflaci\'{o}n m\'{a}s los efectos aleatorios hacen que la estimaci\'{o}n por m\'{a}xima verosimilitud no se resuelve computacionalmente f\'{a}cil, ni mucho menos anal\'{\i}ticamente, ya que requiere la soluci\'{o}n de integrales complejas asociadas a los efectos aleatorios, dichas integrales deben ser aproximadas por diferentes metodolog\'{\i}as, como por ejemplo la cuadratura de Gauss-Hermite, introducido en los modelos lineales generalizados por \cite{Fahrmeir1} o el m\'{e}todo de reducci\'{o}n secuencial \citep{Ogden1}, entre otros. El hecho de que la estimaci\'{o}n v\'{\i}a m\'{a}xima verosimilitud sea complicada, hace que otros autores como \cite{Galvis1} busquen una soluci\'{o}n alternativa dentro de las metodolog\'{\i}as bayesianas, MCMC.\\

A nivel computacional es posible mencionar que existen diferentes paquetes en el software \proglang{R} que describen los datos proporcionales, uno de ellos y el m\'{a}s com\'{u}n es el paquete \pkg{betareg} \citep{Zeileis1}, \citep{Ferrari1} y \citep{Simas1}, el cual incluye la distribuci\'{o}n beta y la estimaci\'{o}n de modelos de regresi\'{o}n beta fijos y mixtos. \cite{Qiu1} implementan el paquete \pkg{simplexreg} el cual incluye la distribuci\'{o}n y el modelo de regresi\'{o}n con efectos fijos de la distribuci\'{o}n simplex.\\

Teniendo en cuenta lo anterior, surge la pregunta de c\'{o}mo se podr\'{\i}a realizar un modelo de regresi\'{o}n mixto para datos proporcionales inflados con ceros y/o unos, sobre las distribuciones beta o simplex, donde la estimaci\'{o}n de sus par\'{a}metros sea v\'{\i}a m\'{a}xima verosimilitud por medio de diferentes variantes de la cuadratura de Gauss-Hermite adaptativa. Debido a esto en este trabajo se incluye en una sola distribuci\'{o}n y en un solo modelo de regresi\'{o}n de efectos fijos y mixtos, las principales distribuciones y modelos de regresi\'{o}n para datos proporcionales inflados con ceros y/o unos, dando como resultado la distribuci\'{o}n ZOIP (Zeros Ones In\-fla\-ted Proporcional), los modelos de regresi\'{o}n ZOIP para efectos fijos y mixtos. La estimaci\'{o}n de los par\'{a}metros regresores de los modelos de regresi\'{o}n ZOIP, son realizados v\'{\i}a m\'{a}xima verosimilitud a trav\'{e}s de las diferentes variaciones de la cuadratura de Gauss-Hermite adaptativa, para esta distribuci\'{o}n y los diferentes modelos de regresi\'{o}n propuestos, se le realizan estudios de simulaci\'{o}n que demuestran la convergencia satisfactoria de sus par\'{a}metros. Adicional a esto, c\'{o}mo no existe un paquete en \proglang{R} que re\'{u}na las principales distribuciones y modelos de regresi\'{o}n de efectos fijos y mixtos para modelar los datos proporcionales inflados con ceros y/o unos, se implementa el paquete \pkg{ZOIP}, que permite generar y ajustar distribuciones y modelos de regresi\'{o}n para efectos fijos y mixtos para datos proporcionales inflados con ceros y/o unos por medio de la metodolog\'{\i}a propuesta. 

\section*{Organizaci\'{o}n del trabajo}

La estructura de este trabajo es considerada de la siguiente manera, en el cap\'{\i}tulo \ref{cap2} se presenta las principales distribuciones descritas en la literatura para datos proporcionales, en diferentes parametrizaciones, adem\'{a}s se implementa la distribuci\'{o}n ZOIP para datos proporcionales inflados con ceros y/o unos, propuesta en este trabajo, luego se muestra c\'{o}mo se utiliza la distribuci\'{o}n ZOIP en el paquete propuesto \pkg{ZOIP}, por \'{u}ltimo se realizan di\-fe\-ren\-tes estudios de simulaci\'{o}n para argumentar la convergencia satisfactoria del ajuste de la distribuci\'{o}n ZOIP y una aplicaci\'{o}n a datos reales.\\

Un modelo de regresi\'{o}n de efectos fijos para datos proporcionales es planteado en el cap\'{\i}tulo \ref{cap3}, con base en la distribuci\'{o}n ZOIP propuesta en el cap\'{\i}tulo anterior, se muestra la inferencia estad\'{\i}stica para la estimaci\'{o}n de los par\'{a}metros v\'{\i}a m\'{a}xima verosimilitud, luego se observa c\'{o}mo utilizar el modelo de regresi\'{o}n para efectos fijos en el paquete \pkg{ZOIP} y las salidas de las funciones de dicho paquete asociadas a la estimaci\'{o}n de un modelo de regresi\'{o}n ZOIP, por \'{u}ltimo se realiza un estudio de simulaci\'{o}n donde se demuestra la convergencia de la estimaci\'{o}n de los par\'{a}metros regresores y se muestra el ajuste de un modelo de regresi\'{o}n ZOIP a datos reales.\\

En el cap\'{\i}tulo \ref{cap4} se muestra la implementaci\'{o}n y el desarrollo anal\'{\i}tico de un modelo de regresi\'{o}n para datos proporcionales inflados con ceros y/o unos, teniendo en cuenta efectos fijos e interceptos aleatorios en los par\'{a}metros de la media y la varianza, se muestra la derivaci\'{o}n de la funci\'{o}n de verosimilitud, necesaria para estimar los par\'{a}metros regresores asociados al modelo, para dicha estimaci\'{o}n es necesario utilizar una aproximaci\'{o}n, la aproximaci\'{o}n utilizada en este trabajo es la cuadratura de Gauss-Hermite, en este cap\'{\i}tulo tambi\'{e}n se muestran las diferentes variaciones que existen en la cuadratura de Gauss-Hermite, seguido de la aproximaci\'{o}n de la funci\'{o}n de verosimilitud por medio de la cuadratura de Gauss-Hermite adaptativa \citep{Liu1}; \citep{Pinheiro1}. Se ilustra la forma de utilizar el paquete \pkg{ZOIP} para el ajuste de un modelo de regresi\'{o}n ZOIP mixto. Luego se muestra el ajuste de un modelo de regresi\'{o}n ZOIP mixto a datos reales y un estudio de simulaci\'{o}n con varios escenarios que nos permite concluir, cu\'{a}l es la mejor metodolog\'{\i}a de estimaci\'{o}n de los par\'{a}metros regresores, en un modelo de regresi\'{o}n mixto para datos proporcionales inflados con ceros y/o unos.\\

Finalmente, en el cap\'{\i}tulo \ref{cap5} se presentan las conclusiones, recomendaciones y trabajos futuros derivados de este trabajo.


 










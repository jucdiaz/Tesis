%\newpage
%\setcounter{page}{1}
\begin{center}
\begin{figure}
\centering%
\epsfig{file=HojaTitulo/EscudoUN.eps,scale=1}%
\end{figure}
\thispagestyle{empty} \vspace*{2.0cm} \textbf{\huge
Modelo de regresi\'{o}n mixto para datos proporcionales inflados con ceros y/o unos}\\[5.5cm]
\Large\textbf{Juan Camilo D\'{\i}az Zapata}\\[5.5cm]
\small Universidad Nacional de Colombia\\
Facultad de Ciencias, Escuela de Estad\'{\i}stica\\
Medell\'{\i}n, Colombia\\
2017\\
\end{center}

\newpage{\pagestyle{empty}\cleardoublepage}

\newpage
\begin{center}
\thispagestyle{empty} \vspace*{0cm} \textbf{\huge
Modelo de regresi\'{o}n mixto para datos proporcionales inflados con ceros y/o unos}\\[3.0cm]
\Large\textbf{Juan Camilo D\'{\i}az Zapata}\\[3.0cm]
\small Tesis presentada como requisito parcial para optar al
t\'{\i}tulo de:\\
\textbf{Magister en Ciencias Estad\'{\i}sticas}\\[2.5cm]
Director:\\
Ph.D. Freddy Hern\'{a}ndez Barajas\\[2.0cm]
L\'{\i}nea de Investigaci\'{o}n:\\
Modelos de Regresi\'{o}n Mixtos\\[2.5cm]
Universidad Nacional de Colombia\\
Facultad de Ciencias, Escuela de Estad\'{\i}stica\\
Medell\'{\i}n, Colombia\\
2017\\
\end{center}

\newpage{\pagestyle{empty}\cleardoublepage}

\newpage
\thispagestyle{empty} \textbf{}\normalsize
\\\\\\%
\textbf{(Dedicatoria)}\\[4.0cm]

\begin{flushright}
\begin{minipage}{12cm}
    \noindent
        \small
				\textsl{Este trabajo es dedicado a mi madre, Blanca\\
				y a mi futura esposa Zaret.}\\
\end{minipage}
\end{flushright}

\newpage{\pagestyle{empty}\cleardoublepage}

\newpage
\thispagestyle{empty} \textbf{}\normalsize
\\\\\\%
\textbf{\LARGE Agradecimientos}
\addcontentsline{toc}{chapter}{\numberline{}Agradecimientos}\\\\

A Dios, por regalarme salud, tiempo, sabidur\'{\i}a e inteligencia para poder realizar este trabajo, que toda la gloria sea para \'{e}l.\\

A mi tutor, Freddy Hern\'{a}ndez, por el constante apoyo, paciencia, orientaci\'{o}n y confianza depositada a lo largo de estos dos a\~{n}os.\\

A mis padres, Blanca y Guillermo por la paciencia, tiempo, cari\~{n}o y apoyo d\'{\i}a tras d\'{\i}a.\\

A mi prometida, Zaret, por su paciencia, apoyo incondicional y comprensi\'{o}n; Por ayudarme, escucharme y alentarme en los momentos m\'{a}s dif\'{\i}ciles y as\'{\i} poder ver realizado este trabajo.\\

A mis sobrinos, Jos\'{e} Manuel y Julieta, por la comprensi\'{o}n y el tiempo regalado, a su corta edad respetaron cada uno de los espacios necesarios para elaborar este trabajo.\\

A mi jefe, Andr\'{e}s Ibarra, por haberme brindado el tiempo para realizar este trabajo y los datos que me permitieron realizar las aplicaciones a datos reales.\\

A la universidad de Antioquia, por brindarme sus instalaciones, que me permitieron trabajar por buen tiempo en este trabajo. 

\newpage{\pagestyle{empty}\cleardoublepage}

\newpage
\textbf{\LARGE Resumen}
\addcontentsline{toc}{chapter}{\numberline{}Resumen}\\\\
El modelo de regresi\'{o}n mixto para datos proporcionales inflados con ceros y/o unos, es un modelo de regresi\'{o}n donde la variable respuesta se encuentra definida a partir de una distribuci\'{o}n para datos proporcionales, como la distribuci\'{o}n beta o la distribuci\'{o}n simplex, que dan resultados en el intervalo cero uno, m\'{a}s dos valores dados por cero y/o uno, representando la ausencia o presencia total de cierta caracter\'{\i}stica. Diferentes autores han trabajado en el desarrollo de diferentes modelos y metodolog\'{\i}as de estimaci\'{o}n, sin embargo, no se ha desarrollado un modelo de regresi\'{o}n mixto para datos proporcionales inflados con ceros y/o unos, que re\'{u}na los principales modelos de regresi\'{o}n de este tipo y que la estimaci\'{o}n de los par\'{a}metros sea v\'{\i}a m\'{a}xima verosimilitud y la cuadratura de Gauss-Hermite. En este trabajo se presenta el paquete \pkg{ZOIP} del sistema computacional \proglang{R}, en el que se implementa la distribuci\'{o}n ZOIP (Zeros Ones Inflated Proporcional), el modelo de regresi\'{o}n para efectos fijos y mixtos, ZOIP, que re\'{u}ne las distribuciones y los modelos de regresi\'{o}n de efectos fijos y mixtos para las distribuciones beta y simplex inflada con ceros y/o unos, la estimaci\'{o}n de los par\'{a}metros se hace v\'{\i}a m\'{a}xima verosimilitud y la cuadratura de Gauss-Hermite. Se realizan diferentes estudios de simulaci\'{o}n que muestran la convergencia de los par\'{a}metros y la alternativa de estimaci\'{o}n que presenta mejor desempe\~{n}o. Adem\'{a}s, se presenta el ajuste de diferentes modelos de regresi\'{o}n ZOIP a datos reales.\\

\textbf{\small Palabras clave:} modelos lineales mixtos, datos proporcionales inflados, cuadratura de Gauss-Hermite, m\'{a}xima verosimilitud.\\


\textbf{\LARGE Abstract}\\\\
The mixed regression model for proportional data inflated with zeros and/or ones is a regression model where the response variable is defined by a proportional data distribution. A proportional data distribution could be the beta distribution or the simplex distribution, which its results are given in the interval zero one, and in other two values given by zero and/or one that represent the absence or total presence of a given characteristic. Although several authors have worked on the development of different models and estimation me\-tho\-do\-lo\-gies, a mixed regression model for proportional data inflated with zeros and/or ones that aggregates the main regression models and the estimation of the parameters through the maximum likelihood and Gauss-Hermite quadrature has not been developed. In this paper, it is presented the ZOIP package for the computational system R, in which the ZOIP (Zeros Ones Inflated Proportional) distribution is implemented. In the ZOIP package is used a regression model for fixed and mixed effects, which unifies the distributions and these types of regressions for beta and simplex distributions inflated with zeros and/or ones, and the estimation of the parameters solved by the maximum likelihood and Gauss-Hermite quadrature. Several simulation studies are performed showing the convergence of the parameters and the estimation alternative that presents the best performance. Furthermore, it is presented the adjustment of different ZOIP regression models to real data.\\

\textbf{\small Keywords:} linear mixed models, proportional inflated data, Gauss-Hermite quadrature, maximum likelihood.\\
%\newpage
%\setcounter{page}{1}
\begin{center}
\begin{figure}
\centering%
\epsfig{file=HojaTitulo/EscudoUN.eps,scale=1}%
\end{figure}
\thispagestyle{empty} \vspace*{2.0cm} \textbf{\huge
Modelo de regresi\'{o}n mixto para datos proporcionales inflados con ceros y/o unos}\\[5.5cm]
\Large\textbf{Juan Camilo D\'{\i}az Zapata}\\[5.5cm]
\small Universidad Nacional de Colombia\\
Facultad de Ciencias, Escuela de Estad\'{\i}stica\\
Medell\'{\i}n, Colombia\\
2017\\
\end{center}

\newpage{\pagestyle{empty}\cleardoublepage}

\newpage
\begin{center}
\thispagestyle{empty} \vspace*{0cm} \textbf{\huge
Modelo de regresi\'{o}n mixto para datos proporcionales inflados con ceros y/o unos}\\[3.0cm]
\Large\textbf{Juan Camilo D\'{\i}az Zapata}\\[3.0cm]
\small Tesis presentada como requisito parcial para optar al
t\'{\i}tulo de:\\
\textbf{Magister en Ciencias Estad\'{\i}sticas}\\[2.5cm]
Director:\\
Ph.D. Freddy Hern\'{a}ndez Barajas\\[2.0cm]
L\'{\i}nea de Investigaci\'{o}n:\\
Modelos de Regresi\'{o}n Mixtos\\[2.5cm]
Universidad Nacional de Colombia\\
Facultad de Ciencias, Escuela de Estad\'{\i}stica\\
Medell\'{\i}n, Colombia\\
2017\\
\end{center}

\newpage{\pagestyle{empty}\cleardoublepage}

\newpage
\thispagestyle{empty} \textbf{}\normalsize
\\\\\\%
\textbf{(Dedicatoria)}\\[4.0cm]

\begin{flushright}
\begin{minipage}{12cm}
    \noindent
        \small
				\textsl{Este trabajo es dedicado a mis padres, Blanca y Guillermo\\
				y a mi futura esposa Zaret.}\\
\end{minipage}
\end{flushright}

\newpage{\pagestyle{empty}\cleardoublepage}

\newpage
\thispagestyle{empty} \textbf{}\normalsize
\\\\\\%
\textbf{\LARGE Agradecimientos}
\addcontentsline{toc}{chapter}{\numberline{}Agradecimientos}\\\\
Esta secci\'{o}n es opcional, en ella el autor agradece a las personas o instituciones que colaboraron en la realizaci\'{o}n de la tesis  o trabajo de investigaci\'{o}n. Si se incluye esta secci\'{o}n, deben aparecer los nombres completos, los cargos y su aporte al documento.\\

\newpage{\pagestyle{empty}\cleardoublepage}

\newpage
\textbf{\LARGE Resumen}
\addcontentsline{toc}{chapter}{\numberline{}Resumen}\\\\
El modelo de regresi\'{o}n mixto para datos proporcionales inflados con ceros y/o unos, es un modelo de regresi\'{o}n donde la variable respuesta se encuentra definida a partir de una distribuci\'{o}n para datos proporcionales, como la distribuci\'{o}n beta o la distribuci\'{o}n simplex, esta variable puede dar resultados de cero o uno, representando la ausencia o presencia total de cierta caracter\'{\i}stica. Diferentes autores han trabajado en el desarrollo de diferentes modelos y metodolog\'{\i}as de estimaci\'{o}n, sin embargo, no se ha desarrollado un modelo de regresi\'{o}n mixto para datos proporcionales inflados con ceros y/o unos, que re\'{u}na los principales modelos de regresi\'{o}n de este tipo y que la estimaci\'{o}n de los par\'{a}metros sea v\'{\i}a m\'{a}xima verosimilitud y la cuadratura de Gauss-Hermite. En este trabajo se presenta el paquete \pkg{ZOIP} del sistema computacional \proglang{R}, en el que se implementa la distribuci\'{o}n ZOIP (Zeros Ones Inflated Proporcional), el modelo de regresi\'{o}n para efectos fijos y mixtos, ZOIP, que re\'{u}ne las distribuciones y los modelos de regresi\'{o}n de efectos fijos y mixtos para las distribuciones beta y simplex inflada con ceros y/o unos, la estimaci\'{o}n de los par\'{a}metros se hace v\'{\i}a m\'{a}xima verosimilitud y la cuadratura de Gauss-Hermite. Se realizan diferentes estudios de simulaci\'{o}n que muestran la convergencia de los par\'{a}metros y la mejor alternativa de estimaci\'{o}n que presenta mejor desempe\~{n}o. Adem\'{a}s, se presenta el ajuste de diferentes modelos de regresi\'{o}n a datos reales.\\

\textbf{\small Palabras clave:} modelos lineales mixtos, datos proporcionales inflados, cuadratura de Gauss-Hermite,  m\'{a}xima verosimilitud.\\


\textbf{\LARGE Abstract}\\\\
El modelo de regresi\'{o}n mixto para datos proporcionales inflados con ceros y/o unos, es un modelo de regresi\'{o}n donde la variable respuesta se encuentra definida a partir de una distribuci\'{o}n para datos proporcionales, como la distribuci\'{o}n beta o la distribuci\'{o}n simplex, esta variable puede dar resultados de cero o uno, representando la ausencia o presencia total de cierta caracter\'{\i}stica. Diferentes autores han trabajado en el desarrollo de diferentes modelos y metodolog\'{\i}as de estimaci\'{o}n, sin embargo, no se ha desarrollado un modelo de regresi\'{o}n mixto para datos proporcionales inflados con ceros y/o unos, que re\'{u}na los principales modelos de regresi\'{o}n de este tipo y que la estimaci\'{o}n de los par\'{a}metros sea v\'{\i}a m\'{a}xima verosimilitud y la cuadratura de Gauss-Hermite. En este trabajo se presenta el paquete \pkg{ZOIP} del sistema computacional \proglang{R}, en el que se implementa la distribuci\'{o}n ZOIP (Zeros Ones Inflated Proporcional), el modelo de regresi\'{o}n para efectos fijos y mixtos, ZOIP, que re\'{u}ne las distribuciones y los modelos de regresi\'{o}n de efectos fijos y mixtos para las distribuciones beta y simplex inflada con ceros y/o unos, la estimaci\'{o}n de los par\'{a}metros se hace v\'{\i}a m\'{a}xima verosimilitud y la cuadratura de Gauss-Hermite. Se realizan diferentes estudios de simulaci\'{o}n que muestran la convergencia de los par\'{a}metros y la mejor alternativa de estimaci\'{o}n que presenta mejor desempe\~{n}o. Adem\'{a}s, se presenta el ajuste de diferentes modelos de regresi\'{o}n a datos reales.\\

\textbf{\small Keywords:} linear mixed models, proportional inflated data, Gauss-Hermite quadrature, maximum likelihood.\\